\documentclass[11pt,a4paper,sans]{moderncv}

\moderncvstyle{banking} 
\moderncvcolor{blue}
\setlength{\hintscolumnwidth}{3.1cm} 
\usepackage[utf8]{inputenc}
\usepackage[scale=0.85]{geometry}
\usepackage{helvet}
\usepackage[french]{babel}
\name{Nicolas}{JEANNE}
\title{Bioinformaticien} 
\address{~~~~~~25, place Dupuy}{31000 Toulouse}{}
\phone[mobile]{06~50~77~65~31}
% \phone[fixed]{05~67~69~04~57}
\email{nicolajea@yahoo.fr}
\extrainfo{Né le 29/09/1976 à Angers.}
 

% changement des puces
\renewcommand{\FrenchLabelItem}{\textbullet}
\renewcommand{\Frlabelitemi}{\scriptsize\color{blue}{$\circ$}}

\begin{document}
\makecvtitle

\section{Compétences}
\cvitem{Programmation}{Python, R, bash, django, JAVA, HTML/CSS, \LaTeX, SQL.}
\cvitem{Versioning}{Git, GitLab, GitHub.}
\cvitem{Pipelines}{Snakemake.}
\cvitem{Conteneurs}{Technologie Singularity.}
\cvitem{Cluster de calcul}{Solutions HPC et slurm.}
\cvitem{Séquençage haut-débit}{Technologies Oxford Nanopore, PacBio, Illumina et 454}
\cvitem{Biologie Moléculaire}{Séquençage Sanger, Séquençage NGS: De Novo, Amplicons, RNA-Seq.}

\section{Expérience professionnelle}
\cventry{depuis 08/2015}{Bioinformaticien}{Virologie - CHU Purpan}{Toulouse}{}
{
\begin{itemize}
\item Mise en place de pipelines Snakemake de diagnostic (Prédiction du tropisme du VIH-1), de recherche (Évolution des quasi-espèces virales, assemblage de Novo de clones viraux) sur données Illumina et d'annotation automatique de génomes de Virus de l'Hépatite E (VHE).
\item Reconstruction de génomes complets du VHE sur données PacBio et Oxford Nanopore.
\item Analyse des insertions/duplications dans la région hyper-variable du VHE.
\item Collaboration au développement de \emph{BD NGS}, outil commun CHU-Oncopole, permettant la gestion des runs NGS et du lancement automatique des pipelines associés.
\item Adaptation et mise en œuvre d'un pipeline de découverte de variants de résistances minoritaires sur des données de séquençage haut-débit des virus de l'hépatite C.
\item Développement de solutions à façon pour le traitement de données biologiques.
\item Développement et adaptation du logiciel PyroVir aux données Illumina et Ion-Torrent.
\end{itemize}}

\cventry{01/2015 - 07/2015}{Stage de bioinformatique}{Laboratoire de Microbiologie et Génétiques Moléculaires - CNRS}{Toulouse}{}{
Étude par RNA-Seq du métabolisme de l'ARN messager dans les opérons : Rôle des BIME chez \textit{E. Coli}.
	\begin{itemize}
	\item Exploration et sélection de données publiques de RNA-Seq.
	\item Mapping et normalisation des données d'expression chez \textit{E. coli}.
	\item Étude statistique de variation de taux d'expression par différentes méthodes (corrélation de profils, segmentation\ldots).
	\item Étude de la stabilité des structures secondaires des BIME par rapport à leurs séquences ancêtres reconstruites.
	\end{itemize}
}

\cventry{08/2010 - 08/2013}{Technicien de biologie moléculaire}{Virologie - CHU Purpan}{Toulouse}{}{
\begin{itemize}
\item Logiciel \textbf{PyroVir} d’analyse de données de pyroséquençage (454) pour la détermination du Tropisme du VIH (\textbf{Brevet}).
\item Étude de variants à partir de données de pyroséquençage amplicons.
\item Mise en œuvre de séquençage haut-débit (technologies Illumina et 454).
\item Étude du tropisme du VIH par séquençage haut-débit.
\item Phylogénétique et recherche de résistances du VIH-1 et des virus des hépatites.
\item Mise en œuvre de séquençage Sanger.
\end{itemize}
}

\clearpage

\cventry{2005 - 07/2010}{Technicien de biologie moléculaire}{Neurogénétique - Hôpital A. Trousseau (APHP)}{Paris}{Centre de référence maladies rares.}{
\begin{itemize}
\item Recherche de mutations par séquençage Sanger dans le cadre du diagnostic de maladies touchant le système nerveux central chez l'enfant
\item Diagnostics pré-nataux.
\item Gestion d'un séquenceur Sanger 96 capillaires.
\item Mise en œuvre d'une technique HRM (High Resolution Melting) pour la détection de la présence de variants. 
\end{itemize}
}

\cventry{2003 - 2005}{Technicien de laboratoire}{Divers laboratoires}{Paris}{}{
\begin{itemize}
\item Biologie moléculaire en endocrinologie. 
\item Virologie.
\end{itemize}
}

\cventry{2002 - 2003}{Technicien d'histocompatibilité}{Institut A. Frappier}{Montréal Canada}{}{
\begin{itemize}
\item Typage HLA. 
\item Gestion de banque de patients en attente de greffe.
\end{itemize}
}

\cventry{2001 - 2002}{Technicien d'immunologie}{Immunologie - Hôpital St. Louis (APHP)}{Paris}{}{
Suivi d'une cohorte de patients sous traitement anti-rétroviral. 
\begin{itemize}
\item Cytométrie de flux.
\item Culture cellulaire.
\end{itemize}
}

\cventry{2000 - 2001}{Technicien d'hématologie}{Hématologie - Hôpital militaire Percy}{Paris}{Service national}{
\begin{itemize}
\item Cytométrie de flux. 
\item Analyses de routine.
\end{itemize}
}

\cventry{1999 - 2000}{Technicien de biologie moléculaire}{Aventis R\&D}{Paris}{}{
\begin{itemize}
\item Projet de séquençage d'un cosmide par méthode shotgun.
\end{itemize}
}

\cventry{1997 - 1999}{Technicien qualité}{Aventis et Sipsy}{Lyon et Angers}{}{
\begin{itemize}
\item Rédaction de procédures qualité.
\item Validation de méthodes.
\end{itemize}
}

\section{Formation}
\cventry{09/2014 - 07/2015}{Master 2 Bioinformatique et Biologie des Systèmes}{Université Paul Sabatier, formation continue}{Toulouse}{\textit{major de promotion}}{
\begin{itemize}
\item Rapport de stage : Étude par RNA-Seq du métabolisme de l'ARN messager dans les opérons, rôle des BIME chez \textit{E. coli}.
\item Revue : Common tools comparison for differential expression analysis : edgeR, DESeq and DESeq2.
\end{itemize}
}

\cventry{09/2013 - 07/2014}{Master 1 Microbiologie Agrobioscience Bioinformatique Biologie des Systèmes}{Université Paul Sabatier, formation continue}{Toulouse}{}{}

\cventry{09/2009 - 07/2012}{Licence Bioinformatique}{CNAM, cours du soir}{Paris}{}{
\begin{itemize}
\item Rapport de stage : PyroTrop, Logiciel d'interprétation de l'analyse du tropisme du VIH-1 par pyroséquençage.
\item Projet tuteuré : Phylog, Logiciel de reconstruction d'arbres phylogénétique, méthodes basées sur les distances.
\item Obtention du Business Language Testing Service (BULATS).
\end{itemize}
}

\cventry{09/2007 - 06/2008}{D.U. de Biologie Moléculaire}{IFTAB, La Pitié Salpétrière}{Paris}{}{}

\cventry{09/1999 - 06/2000}{Licence Professionnelle Microbiologie Industrielle et Biotechnologie}{ESTBA, formation en alternance}{Paris}{}{Alternance effectuée au centre de recherche et développement d'Aventis.}

\cventry{09/1996 - 06/1997}{Licence Professionnelle Assurance et Gestion de la Qualité}{IUT}{Angers}{}{}

\cventry{09/1994 - 06/1996}{DUT de Biologie Appliquée}{IUT}{Angers}{}{Option Analyses Biologiques et Biochimiques.}

\section{Brevets}
\cvitem{PyroVir}{Logiciel d'analyse de données de séquençage haut-débit pour la détermination du tropisme du ~~~VIH-1. Déposé à l'Agence pour la Protection des Programmes via INSERM transfert \linebreak (IDDN.FR.001.160011.000.S.P.2012.000.31230).}

\renewcommand{\refname}{Publications}
\renewcommand{\bibliographyitemlabel}{\arabic{enumiv}.~}
\nocite{*}
% \bibliographystyle mis à 'unsrt' au lieu de 'plain' pour obtenir une chronologie biblio décroisssante
\bibliographystyle{unsrt}
\bibliography{cv_biblio}    

\section{Langues}
\cvitemwithcomment{Anglais}{Lu, parlé, écrit}{BULATS CEF/ALTE Level : C1/4}
\cvitemwithcomment{Espagnol}{Lu, parlé, écrit}{9 mois en Espagne}

\section{Centres d'intérêt}
\cvitem{}{Cinéphile: Bénévole et adhérent sur le projet de création et diffusion de cinéma, \textcolor{blue}{\href{http://www.laforetelectrique.com/}{La forêt électrique}}.}
\cvitem{}{FabLab: Projet Proteios sound, mise en musique des protéines (\textcolor{blue}{\href{https://artilect.fr}{Artilect FabLab}}).}
\cvitem{}{Photographie.}
\cvitem{}{Moto.}

\end{document}